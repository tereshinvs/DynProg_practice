\documentclass[12pt]{article}

\usepackage{a4wide}
\usepackage[utf8]{inputenc}
\usepackage[russian]{babel}
\usepackage{graphicx}
\usepackage{indentfirst}
\usepackage{url}

\usepackage{float}
\usepackage{verbatim}

\usepackage{amsmath}
\usepackage{amsthm}
\usepackage{amssymb}
%\usepackage{wasysym}
%\usepackage{enumitem}
%\usepackage{animate}

\newtheorem{define}{Определение}
\newtheorem{myfact}{Свойство}
\newtheorem{myth}{Теорема}
\newtheorem{assumption}{Предположение}

%\setlist{nolistsep, itemsep=0cm, parsep=0pt}

\newcommand{\ol}[1]{\overline{#1\vphantom{()}}}
\newcommand{\norm}[1]{\left\lVert #1 \right\rVert}
\newcommand*{\hm}[1]{#1\nobreak\discretionary{}{\hbox{$\mathsurround=0pt #1$}}{}}
\newcommand{\scalar}[2]{\left<#1,#2\right>}
\newcommand{\const}{\ensuremath{\operatorname{const}}}
\newcommand{\sgn}{\ensuremath{\operatorname{sgn}}}
\renewcommand{\d}[1]{\ensuremath{\operatorname{d}\!{#1}}}
\newcommand\abs[1]{\left\lvert #1 \right\rvert} % модуль
\newcommand\brackets[1]{\left( #1 \right)} % скобки
\newcommand{\R}{\ensuremath{\mathbb{R}}} % R - мн-во вещественных чисел
\newcommand{\N}{\ensuremath{\mathbb{N}}} % N - мн-во натуральных чисел
\newcommand{\Z}{\ensuremath{\mathbb{Z}}} % Z - мн-во целых чисел
\renewcommand{\C}{\ensuremath{\mathbb{C}}} % C - мн-во комплексных чисел
\newcommand{\E}{\ensuremath{\mathcal{E}}} % E --- эллипсоид
%\newcommand{\norm}[1]{\left\lVert #1 \right\rVert} % норма
\DeclareMathOperator*{\thus}{\Rightarrow} % следствие с возможностью использовать limits
\DeclareMathOperator*{\tolim}{\to} % стремление с возможностью использовать limits
\DeclareMathOperator*{\Argmax}{Argmax} % Argmax с возмножностью использовать limits
\DeclareMathOperator{\rank}{rank} % ранг
\DeclareMathOperator{\e}{e} % экспонента


\begin{document}
\thispagestyle{empty}

\begin{center}
\ \vspace{-3cm}

\includegraphics[width=0.5\linewidth]{msu}\\

{\scshape Московский государственный университет имени М.~В.~Ломоносова}\\
Факультет вычислительной математики и кибернетики\\
Кафедра системного анализа
\vfill

{\LARGE Отчетная работа по практикуму}

\vspace{1cm}
{\Huge\bfseries <<Построение информационного множества динамической системы>>}

~\\
~\\
~\\

\begin{flushright}
  \large
  \textit{Студент 515 группы}\\
  В.С.Терёшин

 \vspace{5mm}
  ~\\
  ~\\

  \textit{Руководитель практикума}\\
   И.\,В.~Востриков

\end{flushright}

\end{center}

\vfill

\begin{center}
Москва, 2015
\end{center}

\newpage

\tableofcontents

\newpage

\section{Постановка задачи}
Построить внешнюю оценку информационного множества для системы:
$$\dot{x}(t) = A(t)x(t) + C(t)v(t)$$
при уравнение наблюдения измерительного устройства:
$$y(t) = G(t)x(t) + \xi$$
на отрезке $t \in [t_0;t_1]$.

Пусть $v \in \E(q(t),Q(t))$, а $\xi \in \E(0, R(t))$, где $Q = Q' > 0$, $R = R' > 0$. А $x(t_0) \in \mathcal{X}^0 = \E(x^0, X^0)$. Необходимо построить соответствующую трубку информационного множества.

\section{Теория}
Для начала введем понятие эллипсоида над пространством $\R^n$. Пусть $q \in \R^n$ --- координата центра эллипсоида и $Q \in \R^{n \times n}, Q = Q' > 0$ --- матрица конфигурации соответственно.
\begin{define}
Эллипсоидом $\E(q,Q)$ называется множество
$$
\E(q,Q) = \left\{ x \in \R^n \mid \; <x-q,Q^{-1}(x-q)> \leqslant 1 \right\}.
$$
\end{define}

Так же введем второе определение, основанное на понятие опорной функции эллипсоида, смягчая требования на матрицу конфигурации до вида $Q = Q' \geqslant 0$.
\begin{define}
Эллипсоидом $\E(q,Q)$ называется множество
$$
\E(q,Q) = \left\{ x \in \R^n \mid \; <l,x> \leqslant <l,q> + <l,Ql>^{1/2}, l \in \R^n \right\}.
$$
\end{define}

\begin{define}
Множеством достижимости системы в момент времени $t$ называется множество
$$
\mathcal{X}[t] = \left\{ x(t; t_0, x_0) \mid \;  \exists x_0 \in \mathcal{X}^0  \right\},
$$
где $x(t)$ --- решение данной системы.
\end{define}
\begin{define}
Трубкой достижимости системы называется многозначное отображение
$$
\mathbb{X}[t] = \left\{ (\tau,x) \in \R \times \R^n \mid \;  x \in \mathcal{X}[\tau], \tau \in [t_0,t] \right\}.
$$
\end{define}
В силу уравнения измерительного устройства
$$y[t] = G(t)x(t) + \xi(t)$$
получим, что
$$G(t)x(t) \in y[t] + \E(0,R(t)).$$
Введем обозначения
\begin{gather}
k^2(t,x) = <y[t] - G(t)x,R^{-1}(y[t] - G(t)x)>, \notag \\
\chi(t) = 1 \text{, если } k^2(t,x) - 1 > 0, \notag \\
\chi(t) = 0 \text{, если } k^2(t,x) - 1 \leqslant 0. \notag
\end{gather}
\begin{define}
Определим информационное множество $\mathcal{X}$ следующим образом
$$\mathcal{X} = {x: V(t,x) \leqslant 0}$$
где $V(t,x)$ --- функция цены, заданная следующим образом
$$V(t,x) = \min\limits_v\left\{\phi_0(x) + \int_{t_0}^{t}(k^2(s,x(s)) - 1)_{+}ds|x(t) = x)\right\}, \; t \leqslant t_0$$
$$\phi_0(x) = (<x - x^0,(X^0)^{-1}(x - x^0))>)_{+}$$
\end{define}
Уравнение ГЯБ:
$$V_t + \textbf{H}(t,x,V_x) = 0, V(t_0,x) = d^2(x,\mathcal{X}^0),$$
где
$$\textbf{H}(t,x,V_x) = max{<p, f(t,x,v)>|v \in \E(q,Q)} - d^2(y(t) - g(t),\E(0,R))$$
Будем искать внешние аппроксимации множества $\mathcal{X}$ из следующего:
\begin{assumption}
Заданными функциями $\textit{H}(t,x,V_x)$, $w^+(t,x) \in C^1$ и $\mu(t)\in L_1$
можно оценить $\textbf{H}(t,x,V_x)$ следующим неравенством
$$\textbf{H}(t,x,V_x) \leqslant \textit{H}(t,x,V_x), {t,x,p}$$
$$w^{+}(t,x) + \textit{H}(t,x,w^+_x)\leqslant \mu(t)$$
Тогда соответствующая оценка нашего информационного множества, будет содержать само множество:
$$\mathcal{X} \subseteq \mathcal{X}^+$$
\end{assumption}
 Для линейной системы гамильтониан имеет следующий вид:
 $$\textbf{H}(t,x,V_x) = max{<p, A(t)x + C(t)v>|v \in \E(q,Q)} - \chi(t)(k^2(t,x) - 1)$$
 И, используя соотношение
$$<p,C(t)\textit{Q}(t)C'(t)p>^{1/2} \leqslant \gamma^2(t) + (4\gamma^2(t))^{-1}<p,C(t)\textit{Q}(t)C'(t)p>, \forall p \in \R^n$$
получаем
$$\textit{H}(t,x,p) = <p, A(t)x + C(t)q> + \gamma^2(t) + (4\gamma^2(t))^{-1}<p,C(t)Q(t)C'(t)p>  - \chi(t)(k^2(t,x) - 1)$$
$$\textbf{H}(t,x,V_x) \leqslant \textit{H}(t,x,V_x).$$
Будем искать оценку вида:
$$w(t,x) = <x - x*(t), P(t)(x - x^*(t))> + h^2(t) - 1,$$
где $h^2(t,x)$ --- дифференцируема, $P(t) = P'(t) > 0$ и предположение $k^2(t,x) \leqslant 1$, так что $\chi(t) = 1$.
В силу неравенств и оценок получим теорему:
\begin{myth}
Внешней оценкой информационного множества $\mathcal{X}$ будет эллипсоид:
$$\E(x^*(t),P^{-1}(t)\beta(t))$$
где $P(t), x^*(t), \beta(t)$ определяются из следующих дифференциальных уравнений.
$$\dot{P} + P'A(t) + A'(t)P + (\gamma^2(t))^{-1}PC(t)Q(t)C'(t)P - \chi(t)G'(t)R(t)G(t) = 0$$
$$\dot{x}^* = A(t)x^*(t) + C(t)q(t) + \chi(t)P^{-1}(t)G'(t)R(t)(y(t) - G(t)x^*(t))$$
$$\dot{h}^2 = <y(t) - G(t)x^*(t), R(t)(y(t) - G(t)x^*(t))>$$
при начальных условиях
$$P(t_0) = (X^0)^{-1}, x^*(t_0) = x^0, h^2(t,x) = 0,$$
а  $\beta(t)$ получается из следующего равенства:
$$\beta(t) = 1 - h^2(t,x) + \int_{t_0}^t(\gamma^2(s) + 1)ds$$
\end{myth}
\begin{myth}
$\mathcal{X}[t] = \cap \left\{\E_{\gamma}[t]|\gamma^2(\cdot)\right\}$
\end{myth}
\section{Примеры}
\subsection{Пример 1}
\begin{gather}
A = \left[ \begin{array}{cc}
0 & 0\\
0 & -1
\end{array} \right],
C = \left[ \begin{array}{cc}
1 \\
0.5
\end{array} \right], \notag \\
G_1 = \left[ \begin{array}{cc}
1 & 1\\
0 & -1
\end{array} \right],
G_2 = \left[ \begin{array}{cc}
0 & 0\\
-1 & 1
\end{array} \right],
G_3 = \left[ \begin{array}{cc}
-1 & 1\\
0 & 1
\end{array} \right], \notag\\
R = 1, q = 1, Q = 1, t_0 = 1, t_1 = 4, \notag\\
x_0 = \left[ \begin{array}{cc}
0\\
0
\end{array} \right],
X_0 = \left[ \begin{array}{cc}
1 & 0\\
0 & 1
\end{array} \right], \notag \\
\gamma_1(t) = 0.9 t^2, \gamma_2(t) = 1/t^2, \gamma_3(t) = 1.5/t. \notag
\end{gather}

Трубки информационного множества, соответствующий измерениям $G_1, G_2, G_3$ (соответственно зелёным, красным и синим):

\includegraphics[scale=1]{pics/ex1_tube_diff.eps}

Трубка информационного множества, соответствующее измерению $G_2$:

\includegraphics[scale=1]{pics/ex1_tube_2.eps}

Информационное множество для измерения $G_1$ в момент времени $t_1$:

\includegraphics[scale=1]{pics/ex1_approx.eps}

\subsection{Пример 2}
\begin{gather}
A = \left[ \begin{array}{cc}
1 & 2\\
-2 & -1
\end{array} \right],
C = \left[ \begin{array}{cc}
1 \\
-1
\end{array} \right], \notag \\
G_1 = \left[ \begin{array}{cc}
1 & 1\\
0 & -1
\end{array} \right],
G_2 = \left[ \begin{array}{cc}
1 & 1\\
-1 & 1
\end{array} \right],
G_3 = \left[ \begin{array}{cc}
-1 & 1\\
0 & 1
\end{array} \right], \notag\\
R = 1, q = 1, Q = 1, t_0 = 1, t_1 = 4, \notag\\
x_0 = \left[ \begin{array}{cc}
0\\
0
\end{array} \right],
X_0 = \left[ \begin{array}{cc}
1 & 0\\
0 & 1
\end{array} \right], \notag \\
\gamma_1(t) = 0.9 t^2, \gamma_2(t) = 1/t^2, \gamma_3(t) = 1.5/t. \notag
\end{gather}

Трубки информационного множества, соответствующий измерениям $G_1, G_2, G_3$ (соответственно зелёным, красным и синим):

\includegraphics[scale=1]{pics/ex2_tube_diff.eps}

\subsection{Пример 3}
\begin{gather}
A = \left[ \begin{array}{cc}
1 & 2\\
-2 & -1
\end{array} \right],
C = \left[ \begin{array}{cc}
1 \\
-1
\end{array} \right], \notag \\
G_1 = \left[ \begin{array}{cc}
1 & 1\\
0 & -1
\end{array} \right],
G_2 = \left[ \begin{array}{cc}
1 & 1\\
-1 & 1
\end{array} \right],
G_3 = \left[ \begin{array}{cc}
-1 & 1\\
0 & 1
\end{array} \right], \notag\\
R = 1, q = 1, Q = 2, t_0 = 1, t_1 = 4, \notag\\
x_0 = \left[ \begin{array}{cc}
0\\
0
\end{array} \right],
X_0 = \left[ \begin{array}{cc}
1 & 2\\
-2 & 3
\end{array} \right], \notag \\
\gamma_1(t) = 0.9 t^2, \gamma_2(t) = 1/t^2, \gamma_3(t) = 1.5/t. \notag
\end{gather}

Трубки информационного множества, соответствующий измерениям $G_1, G_2, G_3$ (соответственно зелёным, красным и синим):

\includegraphics[scale=1]{pics/ex3_tube_diff.eps}

\begin{comment}
\subsection{Пример №1}
Пусть $A = 
\begin{bmatrix} 0 & 1 \\ -5 & 0 \end{bmatrix}$,
$C =  \begin{bmatrix} 0 \\ 0 \end{bmatrix},$ $G = \begin{bmatrix} 1 & 0 \\ 0 & 1 \end{bmatrix},$
$q = 1$, $t \in [2; 3]$, $x_0 = (0,0)'$
~\\
Функции $\gamma$:

$\gamma_1(t) = 0.5$, $\gamma_2(t) = 1$, $\gamma_3(t) = 1.5$
~\\

\begin{figure}[H]
\begin{center}
\includegraphics[width=0.7\textwidth]{set1}
\caption{Информационное множество.Красным выделено информационное множество.}
\end{center}
\end{figure}
\begin{figure}
\begin{center}
  \includegraphics[width=0.8\textwidth]{tube11}
  \includegraphics[width=0.8\textwidth]{tube12}
  \caption{Трубка достижимости}
\end{center}
\end{figure}

При $G = \begin{bmatrix} 1 & 1 \\ 0 & -1 \end{bmatrix}$

\begin{figure}[H]
\begin{center}
\includegraphics[width=0.7\textwidth]{set2}
\caption{Информационное множество.Красным выделено информационное множество.}
\end{center}
\end{figure}
\begin{figure}
\begin{center}
  \includegraphics[width=0.8\textwidth]{tube21}
  \includegraphics[width=0.8\textwidth]{tube22}
  \caption{Трубка достижимости}
\end{center}
\end{figure}

При $G = \begin{bmatrix} 1 & 0 \\ 0 & 1 \\ 1 & 1\end{bmatrix}$
\begin{figure}[H]
\begin{center}
\includegraphics[width=0.7\textwidth]{set3}
\caption{Информационное множество.Красным выделено информационное множество.}
\end{center}
\end{figure}
\begin{figure}
\begin{center}
  \includegraphics[width=0.8\textwidth]{tube31}
  \includegraphics[width=0.8\textwidth]{tube32}
  \caption{Трубка достижимости}
\end{center}
\end{figure}

\subsection{Пример №2}
Пусть $A =
\begin{bmatrix} 1 & 0 \\ 0 & 1 \end{bmatrix}$,
$C =  \begin{bmatrix} 1 \\ 0.5 \end{bmatrix},$ $G = \begin{bmatrix} 1 & 0 \\ 0 & 1 \end{bmatrix},$
$q = 1$, $t \in [2; 3]$, $x_0 = (0,0)'$
~\\
Функции $\gamma$:

$\gamma_1(t) = 0.9t^2$, $\gamma_2(t) = 1/t^2$, $\gamma_3(t) = 1.5/t$
~\\
\begin{figure}[H]
\begin{center}
\includegraphics[width=0.7\textwidth]{set4}
\caption{Информационное множество.Красным выделено информационное множество.}
\end{center}
\end{figure}
\begin{figure}
\begin{center}
  \includegraphics[width=0.8\textwidth]{tube41}
  \includegraphics[width=0.8\textwidth]{tube42}
  \caption{Трубка достижимости}
\end{center}
\end{figure}

При $G = \begin{bmatrix} 1 & 1 \\ 0 & -1 \end{bmatrix}$

\begin{figure}[H]
\begin{center}
\includegraphics[width=0.7\textwidth]{set5}
\caption{Информационное множество.Красным выделено информационное множество.}
\end{center}
\end{figure}
\begin{figure}
\begin{center}
  \includegraphics[width=0.8\textwidth]{tube51}
  \includegraphics[width=0.8\textwidth]{tube52}
  \caption{Трубка достижимости}
\end{center}
\end{figure}

\subsection{Пример №3}
Пусть $A =
\begin{bmatrix} 0 & 0 \\ 0 & -1 \end{bmatrix}$,
$C =  \begin{bmatrix} 1 \\ 0.5 \end{bmatrix},$ $G = \begin{bmatrix} 1 & 1 \\ 0 & -1 \end{bmatrix},$
$q = 1$, $t \in [2; 3]$, $x_0 = (0,0)'$
~\\
Функции $\gamma$:

$\gamma_1(t) = 0.9t^2$, $\gamma_2(t) = 1/t^2$, $\gamma_3(t) = 1.5/t$
~\\
\begin{figure}[H]
\begin{center}
\includegraphics[width=0.7\textwidth]{set6}
\caption{Информационное множество.Красным выделено информационное множество.}
\end{center}
\end{figure}
\begin{figure}
 \begin{center}
  \includegraphics[width=0.8\textwidth]{tube61}
  \includegraphics[width=0.8\textwidth]{tube62}
  \caption{Трубка достижимости}
  \end{center}
\end{figure}
\subsection{Пример №4}
Пусть $A =
\begin{bmatrix} 1 & 0 \\ 0 & 1 \end{bmatrix}$,
$C =  \begin{bmatrix} 0 \\ 1 \end{bmatrix},$ $G = \begin{bmatrix} 1 & 0 \\ 0 & 0 \end{bmatrix},$
$q = 1$, $t \in [2; 3]$, $x_0 = (0,0)'$
~\\
Функции $\gamma$:

$\gamma_1(t) = 0.9t^2$, $\gamma_2(t) = 1/t^2$, $\gamma_3(t) = 1.5/t$
$$y_1 = x + 0.5, y_2 = 0.5x^2$$
\begin{figure}[H]
\begin{center}
\includegraphics[width=0.7\textwidth]{vs}
\caption{Применение интегрального метода для разных измерений}
\end{center}
\end{figure}
\end{comment}
\newpage
\begin{thebibliography}{99}
\bibitem{newllang} \textit{A.B. Kurzhanski, Pravin Varaiya} "Dynamic and Control of Trajectory Tubes", Birkhauser.
\end{thebibliography}
\end{document} 