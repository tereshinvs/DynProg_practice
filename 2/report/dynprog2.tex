\documentclass[12pt]{article}

\usepackage{a4wide} % уменьшает поля
\usepackage[utf8]{inputenc}
\usepackage[russian]{babel} % включает русский язык
\usepackage{graphicx} % позволяет подключить .eps - файлы
\usepackage{amsmath}
\usepackage{amsthm} % теоремы от AMS
\usepackage{amssymb} % для работы с математическими R и проч.
\usepackage{floatrow}
\usepackage{mathrsfs}
\usepackage[section]{placeins}
\usepackage{indentfirst} % абзац после заголовка
\usepackage{misccorr} % точки в заголовках

%\graphicspath{{pics/}}


%\newtheoremstyle{rusdef}
%  {3pt}% measure of space to leave above the theorem. E.g.: 3pt
%  {3pt}% measure of space to leave below the theorem. E.g.: 3pt
%  {\itshape}% name of font to use in the body of the theorem
%  {\parindent}% measure of space to indent
%  {\bfseries}% name of head font
%  {.}%
%  {.5em}%
%  {}
   
  
\theoremstyle{rusdef}
\renewcommand\qedsymbol{$\blacksquare$}
\newtheorem{remark}{Замечание}
\newtheorem{theorem}{Теорема}
\newtheorem{definition}{Определение}
\newtheorem{proposition}{Утверждение}

\newcommand*{\hm}[1]{#1\nobreak\discretionary{}{\hbox{$\mathsurround=0pt #1$}}{}}
\newcommand{\scalar}[2]{\left<#1,#2\right>}
\newcommand{\const}{\ensuremath{\operatorname{const}}}
\newcommand{\sgn}{\ensuremath{\operatorname{sgn}}}
\renewcommand{\d}[1]{\ensuremath{\operatorname{d}\!{#1}}}
\newcommand\abs[1]{\left\lvert #1 \right\rvert} % модуль
\newcommand\brackets[1]{\left( #1 \right)} % скобки
\newcommand{\R}{\ensuremath{\mathbb{R}}} % R - мн-во вещественных чисел
\newcommand{\N}{\ensuremath{\mathbb{N}}} % N - мн-во натуральных чисел
\newcommand{\Z}{\ensuremath{\mathbb{Z}}} % Z - мн-во целых чисел
\renewcommand{\C}{\ensuremath{\mathbb{C}}} % C - мн-во комплексных чисел
\newcommand{\E}{\ensuremath{\mathcal{E}}} % E --- эллипсоид
\newcommand{\norm}[1]{\left\lVert #1 \right\rVert} % норма
\DeclareMathOperator*{\thus}{\Rightarrow} % следствие с возможностью использовать limits
\DeclareMathOperator*{\tolim}{\to} % стремление с возможностью использовать limits
\DeclareMathOperator*{\Argmax}{Argmax} % Argmax с возмножностью использовать limits
\DeclareMathOperator{\rank}{rank} % ранг
\DeclareMathOperator{\e}{e} % экспонента

% dot above minus
\makeatletter
\newcommand{\dotminus}{\mathbin{\text{\@dotminus}}}
\newcommand{\@dotminus}{%
	\ooalign{\hidewidth\raise1ex\hbox{.}\hidewidth\cr$\m@th-$\cr}%
}
\makeatother
% end dot above minus

\newcommand{\rpm}{\sbox0{$1$}\sbox2{$\scriptstyle\pm$}
\raise\dimexpr(\ht1)/2\relax\box2 } % крутой плюс-минус

\begin{document}
\thispagestyle{empty}

\begin{center}
\ \vspace{-3cm}

\includegraphics[width=0.5\textwidth]{msu.eps}\\

{\scshape Московский Государственный Университет имени М.~В.~Ломоносова}\\
Факультет Вычислительной Математики и Кибернетики\\
Кафедра Системного Анализа
\vfill

{\LARGE Отчёт по практикуму по курсу <<Внешние оценки множества	разрешимости>>}
\vspace{.5cm}

\end{center}

\vspace{1cm}

\begin{flushright}
\large
\textit{Студент 415 группы}\\
В.~С.~Терёшин\\
\vspace{5mm}
%\textit{Руководитель практики}\\
%акад., проф., А. Б. Куржанский

\end{flushright}

\vfill
\begin{center}
{\large
Москва, 2014г.}
\end{center}

\newpage
\tableofcontents
\newpage
\section{Постановка задачи}
Дана система с управлением и неопределённостью:
\begin{gather}
\left\{
\begin{aligned}
\dot{x} = A(t)x(t) + B(t)u(t) + C(t)v(t), \; t \in [t_0, t_1] \\
x(t_1) \in \E(x_1, X_1), \\
u(t) \in \E(p(t), P(t)), \\
v(t) \in \E(q(t), Q(t)), \\
x(t) \in \R^n, u(t) \in \R^m, v(t) \in \R^k.
\end{aligned}
\right.
\end{gather}
Построить внешние эллипсоидальные оценки множества разрешимости для данной системы и изобразить полученные оценки следующими способами:
\begin{enumerate}
\item в проекции на двумерную плоскость;
\item в проекции на двумерную динамическую плоскость;
\item проекцию трубки достижимости на трехмерную плоскость;
\item проекцию трубки достижимости на трехмерную динамическую плоскость.
\end{enumerate}

\section{Теория}
\begin{definition}
Для системы (1)
\begin{enumerate}
\item Множество достижимости $\mathbb{X}(t_0 , t_1 , X_0 )$ определяется как множество всех возможных состояний $x_1$ системы в момент времени $t_1$ таких, что для каждого элемента $x_0 \in X_0$ существует некоторое управление $u(t)$ такое, что для любой неопределённости $v(t)$ $x_0$ переводится в это множество.
\item Множество разрешимости $W(t_0 , t_1 , M)$ в момент времени $t_0$ определяется так:
$$
W(t_0 , t_1 , M) = {x_0 : \mathbb{X}(t_0, t_1, x_0) \cap M \neq \varnothing}.
$$
\end{enumerate}
\end{definition}

\begin{definition}
Трубкой разрешимости $W[t]$ системы (1) называется многозначное отображение
$$
W[t] = W(t, t_1, M).
$$
\end{definition}

\begin{definition}
Эллипсоид с центром $q$ и матрицей $Q$ определяется как
$$
\E(q, Q) = {x: \scalar{x-q}{Q^{-1}(x-q)} \leqslant 1}.
$$
Опорная функция в данном случае равна
$$
\rho(l \mid \E(q,Q)) = \scalar{l}{q} + \scalar{l}{Ql}^{1/2}.
$$
\end{definition}

\begin{definition}
Суммой Минковского $k$ компакных множеств $\mathcal{X}_k \subseteq \R^n$ называется
$$
\mathcal{X}_1 \oplus \ldots \oplus \mathcal{X}_k = \bigcup\limits_{x_1 \in \mathcal{X}_1} \ldots \bigcup\limits_{x_k \in \mathcal{X}_k} \left\{ x_1 + \ldots + x_k \right\}.
$$
\end{definition}

\begin{definition}
Для компактных множеств $\mathcal{X}_1, \mathcal{X}_2 \subseteq \R^n$ разностью Минковского называется
$$
\mathcal{X}_1 \dotminus \mathcal{X}_2 = \left\{ x \in \R^n \mid x + \mathcal{X}_2 \subseteq \mathcal{X}_1 \right\}.
$$
\end{definition}

\begin{proposition}
Пусть $\E(q, Q) \subseteq \R^n$, $A \in \R^{m \times n}$ и $b \in \R^m$. Тогда
$$
A\E(q, Q) + b = \E(Aq + b, AQA^T).
$$
\end{proposition}

\begin{proposition}
Пусть $l \in \R^n$, $l \neq 0$. Тогда внешней аппроксимацией $\E(q, Q_l^+)$ суммы по Минковскому $\E(q_1, Q_1) \oplus \ldots \oplus \E(q_k, Q_k)$ называется эллипсоид такой, что
$$
\E(q_1, Q_1) \oplus \ldots \oplus \E(q_k, Q_k) \subseteq \E(q, Q_l^+)
$$
и
$$
\rho(\pm l \mid \E(q_1, Q_1) \oplus \ldots \oplus \E(q_k, Q_k)) = \rho(\pm l \mid \E(q, Q_l^+)).
$$
В этом случае
\begin{gather}
q = q_1 + \ldots + q_k, \notag \\
Q_l^+ = \left( \scalar{l}{Q_1 l}^{1/2} + \ldots + \scalar{l}{Q_k l}^{1/2} \right) \left( \frac{1}{\scalar{l}{Q_1 l}^{1/2}}Q_1 + \ldots + \frac{1}{\scalar{l}{Q_k l}^{1/2}}Q_k \right) \notag
\end{gather}
\end{proposition}

\begin{proposition}
Вняшняя аппроксимация $\E(q, Q_l^+)$ для разности Минковского $\E(q_1, Q_1) \dotminus \E(q_2, Q_2)$ имеет следующие центр и матрицу конфигурации:
\begin{gather}
q = q_1 - q_2, \notag \\
Q_l^+ = \left( Q_1^{1/2} - SQ_2^{1/2} \right)^T \left( Q_l^{1/2} - SQ_2^{1/2} \right), \notag
\end{gather}
где $S$ --- ортогональная матрица такая, что $Q_1^{1/2}l \parallel SQ_2^{1/2}l$.
\end{proposition}
\begin{remark}
Не для каждого направления $l$ существует внешняя аппроксимация для разности Минковского касается этой разности.
\end{remark}

\begin{proposition}
Внешняя аппроксимация $\E(q, Q_l^+)$ для выражения $\E(q_0, Q_0) + \int\limits_{t_0}^t \E(q(t), Q(t))$ иммеет следующие центр и матрицу конфигурации:
\begin{gather}
\dot{q^+} = q(t), \; q^+(t_0) = q_0, \notag \\
\dot{Q^+} = \pi(t) Q^+(t) + \pi^{-1}(t) Q(t), \; Q^+(t_0) = Q_0, \notag
\end{gather}
где
$$
\pi(t) = \frac{\scalar{l}{Q(t)l}^{1/2}}{\scalar{l}{Q_0 l}^{1/2} + \int\limits_{t_0}^t \scalar{l}{Q(\tau)l}^{1/2} d\tau}.
$$
\end{proposition}

\begin{definition}
Матрица Коши $X(t, \tau)$ для системы (1) определяется как решение системы
$$
\left\{
\begin{aligned}
\frac{\partial}{\partial t} X(t, \tau) = -A(t)X(t, \tau), \\
X(\tau, \tau) = E.
\end{aligned}
\right.
$$
\end{definition}

Решение системы (1) с краевым условием $x(t_1) = x_1$ и управлением $u(t)$ записывается в виде
$$
x(t) = X(t, t_1)x_1 + \int\limits_{t_1}^{t} X(t, \tau)B(\tau)u(\tau)d\tau + \int\limits_{t_1}^{t} X(t, \tau)C(\tau)v(\tau)d\tau.
$$

\subsection{Внешняя эллипсоидальная аппроксимация}
Основное утверждение, на котором основывается построение внешней эллипсоидальной аппроксимации:
\begin{proposition}
Пусть даны эллипсоиды $\E_k = \E(0, Q_k )$, $k = 1 \ldots n$. Тогда для любых $r_k > 0$, $k = 1\ldots n$ для $Q_+ = \sum\limits_{k=1}^n r_k \sum\limits_{k=1}^n \frac{Q_k}{r_k}$ выполнено включение
$$
\E_+ = \E(0, Q_+) \subseteq \sum\limits_{k=1}^n \E_k.
$$
\end{proposition}

Множество достижимости можно приблизить выражением
%\begin{gather}
%W[t] = \left( X(t, t_1) \E(x_1, X_1) \dotminus \int\limits_{t}^{T_1} X(t, \tau) C(\tau) \E(q(\tau), Q(\tau)) d\tau \right) \oplus \notag \\
%\oplus \int\limits_{t}^{T_1} X(t, \tau) B(\tau) \E(p(\tau), P(\tau)) d\tau. \notag
%\end{gather}
\begin{gather}
W_k[t] = \left( X(t, \tau_k) W_{k-1}[\tau_k] \dotminus \int\limits_{t}^{\tau_k} X(t, \tau) C(\tau) \E(q(\tau), Q(\tau)) d\tau \right) \oplus \notag \\
\oplus \int\limits_{\tau_k}^{t} X(t, \tau) B(\tau) \E(p(\tau), P(\tau)) d\tau, \notag
\end{gather}
$W[t] \sim W_k[t]$ при достаточно больших $k$.

Из определения многозначного интеграла, множество $W[t]$ можно рассматривать как предел интегральных сумм, т.е. как предел сумм эллипсоидов. Строя внешнюю эллипсоидальную аппроксимацию для каждой интегральных суммы, в пределе для $W[t]$ получим аппроксимирующий эллипсоид $\E_+[t] = \E(x_+(t), X_+(t))$, для которого параметры находятся из следующих систем, следующих из свойств, описанных в предыдущем параграфе:
$$
\dot{x_+}(t) = A(t)x_+(t) + B(t)p(t) + C(t)q(t), \; x_+(t_1) = x_1.
$$

Фиксируем какое-либо направление $l_0 \in \R^n$ и считаем $l(t)$ таким, что
$$
\dot{l}(t) = -A^T(t)l(t), \; l(t_0) = l_0.
$$
Тогда
\begin{gather}
\dot{X_l^+}(t) = A(t)X_l^+(t) + X_l^t(t)A^T(t) - \notag \\
- \pi_l(t) X_l^+(t) - \frac{1}{\pi_l(t)} B(t)P(t)B^T(t) + \notag \\
+ (X_l^+(t))^{1/2} S_l(t) (C(t) Q(t) C^T(t))^{1/2} + \notag \\
(C(t) Q(t) C^T(t))^{1/2} S_l^T(t) (X_l^+(t))^{1/2}, \notag \\
X_l^+(t_1) = X_1, \notag
\end{gather}
где
$$
\pi_l(t) = \frac{\scalar{l(t)}{B(t)P(t)B^T(t)l(t)}^{1/2}}{\scalar{l(t)}{X_l^+(t)l(t)}^{1/2}},
$$
а ортогональная матрица $S_l(t)$ удовлетворяет уравнению
$$
S_l(t) (C(t)Q(t)C^T(t))^{1/2} l(t) = \frac{\scalar{l(t)}{C(t)Q(t)C^T(t)l(t)}^{1/2}}{\scalar{l(t)}{X_l^+(t)l(t)}^{1/2}} (X_l^+(t))^{1/2} l(t).
$$

При этом касание аппроксимирующего эллипсоида $\E(x_+(t), X_+(t))$ множества $W[t]$ происходит в точке
$$
x^*(t) = x_+(t) + \frac{X_+(t)l(t)}{\scalar{l(t)}{X_+l(t)}^{1/2}}.
$$

\begin{proof}
%Рассмотрим эволюционное уравнение данной системы:
%$$
%\lim\limits_{\sigma \to 0} \sigma^{-1} h_+\left( \left((I + \sigma A(t)) \mathcal{X}_+(t) \dotminus \sigma C(t)\mathcal{Q}(t)\right) \oplus \sigma B(t)\mathcal{P}(t), \; \mathcal{X}_+(t + \sigma) \right) = 0,
%$$
%что означает, что при малом $\sigma$ для всех $t$ выполнено включение:
%$$
%\left((I + \sigma A(t)) X_+(t) \dotminus \sigma C(t)\mathcal{Q}(t)\right) \oplus \sigma B(t)\mathcal{P}(t) \subseteq \mathcal{X}_+(t + \sigma).
%$$
%Построим внешнюю эллипсоидальную оценку разности по Минковскому в левой части:
%\begin{gather}
%Q^* = \left( \left( (I + \sigma A(t)) X_+(t) (I + \sigma A(t))^T \right)^{1/2} - S(t) \sigma^{1/2} \left( C(t)Q(t)C^T(t) \right)^{1/2} \right)^T \cdot \notag \\
%\cdot \left( \left( (I + \sigma A(t)) X_+(t) (I + \sigma A(t))^T \right)^{1/2} - S(t) \sigma^{1/2} \left( C(t)Q(t)C^T(t) \right)^{1/2} \right), \notag \\
%X_+(t + \sigma) = Q^* + \sigma B(t)P(t)B^T(t) + \pi_l(t)\sigma Q^* + \pi_l^{-1}(t)\sigma^2 B(t)P(t)B^T(t), \notag
%\end{gather}
%где
%$$
%\pi_l(t) = \frac{\scalar{l}{B(t)P(t)B^T(t)}^{1/2}}{\scalar{l}{Q^*l}^{1/2}}.
%$$
Динамика внешней эллипсоидальной аппроксимации заданной системы описывается следующим эволюционным уравнением:
$$
\lim\limits_{\sigma \to 0} h_{-} (W[t-\sigma],((I-\sigma A(t))W(t) \oplus \sigma (-B(t) \mathcal{P}(t))) \dotminus \sigma(-C(t)\mathcal{Q}(t))) = 0, W[t_1] = \E(y_1,Y_1).
$$

Это уравнение рассматривается при эллипсоидальных ограничениях $\mathcal{P}$ и $\mathcal{Q}$ (без ограничения общности будем считать, что их центры расположены в нуле). Фактически оно означает, что при достаточно малых значения $\sigma$ на всем интервале времени $[t_0,t_1]$ выполнено включение
$$
W[t-\sigma] \supseteq ((I-\sigma A(t))W(t) \oplus \sigma (-B(t) \E(0,P(t))) \dotminus \sigma(-C(t)\E(0,Q(t))).
$$

Запишем внешнюю эллипсоидальную аппроксимацию для разности, стоящей в правой части уравнения:
\begin{multline*}
X(t-\sigma) = (1+p^{-1}(t))(I-\sigma A(t))X(t)(I-\sigma A^T(t)) +\\+
(1+p(t)) B(t)P(t)B^T(t) + \sigma^2 C(t)Q(t)C(t) -\\-
[(1+p^{-1}(t))(I-\sigma A(t))X(t)(I-\sigma A^T(t)) +\\+
(1+p(t)) B(t)P(t)B^T(t)]^{1/2}S(t)(C(t)Q(t)C(t))^{1/2} -\\-
(C(t)Q(t)C(t))^{1/2}S^T(t)[(1+p^{-1}(t))(I-\sigma A(t))X(t)(I-\sigma A^T(t)) +\\+
(1+p(t)) B(t)P(t)B^T(t)]^{1/2} =\\=
 \left\{ p(t) = -\pi(t)*\sigma \right\} = \\ =
 X(t) - \sigma\{A(t)X(t) + X(t)A^T(t) - \pi(t)X(t) - \pi^{-1}(t)B(t)P(t)B^T(t) +\\+ (X(t))^{1/2}S(t)(C(t)Q(t)C^T(t))^{1/2} + (C(t)Q(t)C^T(t))^{1/2}S^T(t)(X(t))^{1/2}\} + o(\sigma).
\end{multline*}

Перенеся $Q(t)$ в левую часть, разделив на $\sigma$ и перейдя к пределу, получим указанную выше оценку, касающуюся оригинального множества в направлении $l(t)$ хорошей кривой.

\end{proof}

Внешняя аппроксимация исходного множества разрешимости $W[t]$ естественно определяется как пересечение по всем выбранным векторам $l_0$ эллипсоидов $\E(x_+(t), X_+(t))$, касающихся $W[t]$ в каждый момент времени в плоскости, перпендикулярной к $l(t) \hm= X(t_1,t)l_0$.

Метод нахождения $S_l(t)$ можно найти в \cite{Daryin}.

\section{Описание работы}
При проекции трубки разрешимости на плоскость встаёт вопрос, о том какие именно аппроксимирующие эллипсоиды выбрать, чтобы в момент времени $t_0$ касание между эллипсоидами и точным множеством происходило именно в заданной плоскости (другими словами в момент времени $t_0$ хорошие кривые должны пересекать искомую плоскость). Для того чтобы ответить на этот вопрос мы решаем задачу в прямом времени (в момент $t_0$ эллипсоиды касаются исходного множества по направлениям равномерно распределенным в плоскости) и смотрим куда перейдут начальные направления. Взяв эти новые направления за основу мы решаем задачу в обратном времени, причем к моменту времени $t_0$ новые направления перейдут в исходные и таким образом касание между аппроксимирующим множеством и исходным будет происходить в исходной плоскости.
\section{Результаты работы программы}
\subsection{Пример 1}
Входные данные:
\begin{gather}
A = 0.1 \cdot \left[
\begin{array}{cccc}
1 & 0 & 1 & 0 \\
0 & 1 & 0 & 3 \\
0 & 0 & 2 & 0 \\
0 & 0 & 0 & 1
\end{array}
\right],
B = C = \left[
\begin{array}{cc}
1 & 0 \\
0 & 1 \\
0 & 0 \\
0 & 0
\end{array}
\right], \notag \\
x_1 = \left[
\begin{array}{c}
0 \\
0 \\
0 \\
0
\end{array}
\right],
X_1 = \left[
\begin{array}{cccc}
1 & 0 & 0 & 0 \\
0 & 1 & 0 & 0 \\
0 & 0 & 1 & 0 \\
0 & 0 & 0 & 1
\end{array}
\right], \notag \\
p = \left[
\begin{array}{c}
0 \\
0
\end{array}
\right],
P = \left[
\begin{array}{cc}
1 & 0 \\
0 & 1
\end{array}
\right], \notag \\
q = \left[
\begin{array}{c}
0 \\
0
\end{array}
\right],
Q = \left[
\begin{array}{cc}
1 & 0 \\
0 & 1
\end{array}
\right], \notag \\
T_0 = 0, \; T_1 = 6, \; dT = 0.5, \notag \\
l_1 = \left[
\begin{array}{c}
1 \\
0 \\
0 \\
0
\end{array}
\right],
l_2 = \left[
\begin{array}{c}
0 \\
0 \\
1 \\
0
\end{array}
\right], \notag
\end{gather}

\begin{figure}[p]
	\centering
	\includegraphics[scale=1.0]{pics/plot1.eps}
	\caption{Пример 1: проекция трубки достижимости на $l_1, l_2$}
	\label{pic1_1}
\end{figure}

\begin{figure}[p]
	\centering
	\includegraphics[scale=1.0]{pics/plot2.eps}
	\caption{Пример 1: аппроксимирующие эллипсоиды в момент времени $t = 0$}
	\label{pic1_2}
\end{figure}

\newpage

\subsection{Пример 2}
\begin{gather}
A = 0.1 \sin t \left[
\begin{array}{cccc}
	1 & 0 & 0 & 0 \\
	0 & 1 & 0 & 0 \\
	0 & 0 & 1 & 0 \\
	0 & 0 & 0 & 1
\end{array}
\right],
B = \left[
\begin{array}{cccc}
	1 & 0 & 0 & 0 \\
	0 & 0 & 0 & 0 \\
	0 & 0 & 1 & 0 \\
	0 & 0 & 0 & 0
\end{array}
\right],
C = 0.1 \cdot \left[
\begin{array}{c}
	1 \\
	2 \\
	1 \\
	1
\end{array}
\right], \notag \\
x_1 = \left[
\begin{array}{c}
	0 \\
	0 \\
	0 \\
	0
\end{array}
\right],
X_1 = \left[
\begin{array}{cccc}
	1 & 0 & 0 & 0 \\
	0 & 1 & 0 & 0 \\
	0 & 0 & 1 & 0 \\
	0 & 0 & 0 & 1
\end{array}
\right], \notag \\
p = \left[
\begin{array}{c}
	0 \\
	0 \\
	0 \\
	0
\end{array}
\right],
P = \left[
\begin{array}{cccc}
	20 & 0 & 0 & 0 \\
	0 & 1 & 0 & 0 \\
	0 & 0 & 1 & 0 \\
	0 & 0 & 0 & 0
\end{array}
\right], \notag \\
q = 0,
Q = 0.1, \notag \\
T_0 = 0, \; T_1 = 10, \; dT = 0.5, \notag \\
l_1(t) = \left[
\begin{array}{c}
	1 \\
	0 \\
	0 \\
	0
\end{array}
\right],
l_2(t) = \left[
\begin{array}{c}
	0 \\
	0 \\
	\sin t \\
	\cos t
\end{array}
\right], \notag
\end{gather}

\begin{figure}[p]
	\centering
	\includegraphics[scale=1.0]{pics/plot3.eps}
	\caption{Проекция трубки достижимости на $l_1(t), l_2(t)$}
	\label{pic2_1}
\end{figure}

\begin{figure}[p]
	\centering
	\includegraphics[scale=1.0]{pics/plot4.eps}
	\caption{Аппроксимирующие эллипсоиды в момент времени $t = 0$}
	\label{pic2_2}
\end{figure}

\newpage

\subsection{Пример 3}
Данный пример иллюстирует задачу из задания 1.
\begin{gather}
L = 5, L_1 = 1, L_2 = 3, C_1 = 1, C_2 = 4, U_1^0 = U_2^0 = 0.1, I_1^0 = I_2^0 = 0.2, \tilde{U} = 1, \notag \\
D = L_1 L_2 + L (L_1 + L_2), \notag \\
A = \left[
\begin{array}{cccc}
	0 & 0 & -\frac{L+L_2}{D} & -\frac{L}{D} \\
	0 & 0 & -\frac{L}{D} & -\frac{L+L_1}{D} \\
	\frac{1}{C_1} & 0 & 0 & 0 \\
	0 & \frac{1}{C_2} & 0 & 0
\end{array}
\right],
B = \left[
\begin{array}{cccc}
	\frac{L}{D} \\
	\frac{L+L_1}{D} \\
	0 \\
	0
\end{array}
\right],
C = \left[
\begin{array}{c}
	0 \\
	0 \\
	0 \\
	0
\end{array}
\right], \notag \\
x_1 = \left[
\begin{array}{c}
	I_1^0 \\
	I_2^0 \\
	U_1^0 \\
	U_2^0
\end{array}
\right],
X_1 = 0.01 \cdot \left[
\begin{array}{cccc}
	1 & 0 & 0 & 0 \\
	0 & 1 & 0 & 0 \\
	0 & 0 & 1 & 0 \\
	0 & 0 & 0 & 1
\end{array}
\right], \notag \\
p = 0,
P = \tilde{U}, \notag \\
q = 0,
Q = 1, \notag \\
T_0 = 0, \; T_1 = 30, \; dT = 0.5, \notag \\
l_1 = \left[
\begin{array}{c}
	0 \\
	0 \\
	0 \\
	1
\end{array}
\right],
l_2 = \left[
\begin{array}{c}
	0 \\
	0 \\
	1 \\
	0
\end{array}
\right], \notag
\end{gather}

\begin{figure}[p]
	\centering
	\includegraphics[scale=1.0]{pics/plot5.eps}
	\caption{Проекция трубки достижимости на $l_1, l_2$}
	\label{pic3_1}
\end{figure}

\begin{figure}[p]
	\centering
	\includegraphics[scale=1.0]{pics/plot6.eps}
	\caption{Аппроксимирующие эллипсоиды в момент времени $t = 0$}
	\label{pic3_2}
\end{figure}

\newpage
\begin{thebibliography}{99}
	\bibitem{EllToolMan} П.~Гагаринов, А.~А.~Куржанский: Инструкция к Ellipsoidal Toolbox, 2014.
	\bibitem{Daryin} А.~Н.~Дарьин, А.Б.Куржанский: Метод вычисления инвариантных множеств линейных систем большой размерности при неопределённых возмущениях. Доклады РАН. 2012. Т. 446. \textnumero 6. С. 607–611.
\end{thebibliography}

\end{document}